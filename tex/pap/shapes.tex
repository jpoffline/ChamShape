%%\documentclass[a4paper,eqsecnum,twoside,aps,floatfix,notitlepage]{revtex4}
\documentclass[amsmath,amssymb,10pt,eqsecnum, twocolumn]{revtex4}
\usepackage{graphicx,float,subfloat,caption}
\usepackage[breaklinks, colorlinks, citecolor=blue]{hyperref}
\usepackage{bm}% bold math

\usepackage{subfigure}
\usepackage{hyperref}

\input{defs}
 
\begin{document}

\title{Shaping the chameleon}
\author{Jonathan A. Pearson}
\email{j.pearson@nottingham.ac.uk}
\affiliation{School of Physics \& Astronomy, University of Nottingham, Nottingham, NG7 2RD, U.K.}	
\date{\today}

\maketitle




 
 
 
\section{Introduction}

The source density is $\rho(\rbm{x})$.  Different geometrical shapes of this source (such as ellipses, rectangles, triangles, etc) will yield different gravitational potentials, and will alter how the chameleon scalar responds.

The effective potential for the chameleon is
\bea
\qsubrm{V}{eff} = \frac{\Lambda^5}{\phi} + \frac{\rho(\rbm{x})}{M}\phi.
\eea
The equations governing the chameleon scalar and gravitational potential are
\bse
\bea
\nabla^2\phi = \frac{\dd \qsubrm{V}{eff}}{\dd\phi}.
\eea
\bea
\nabla^2\Phi = - \rho(\rbm{x})/\qsubrm{M}{pl}^2
\eea
\ese
The force on a test particle, due to the gravitational and chameleon scalars are given by
\bea
\rbm{F}_{(\phi)} = - \tfrac{1}{M}\nabla\phi,\qquad \rbm{F}_{(\Phi)} = - \nabla\Phi.
\eea
The idea is to obtain the source-shape that maximises the force that a test particle will feel, that cannot be explained by forces of a purely gravitational origin.


\section{Explanation of numerical methods}
Discretize onto a grid. Using successive over relaxation.

By setting
\bea
\phi = \sqrt{M\Lambda^5}\tilde{\phi},
\eea
\bea
\Phi = \frac{M}{\qsubrm{M}{Pl}^2}\sqrt{M\Lambda^5}\tilde{\Phi},
\eea
\bea
x^{\mu} = \sqrt{M}\left(M\Lambda^5\right)^{1/4}\tilde{x}^{\mu},
\eea
the equations become
\bea
\tilde{\nabla}^2\tilde{\phi} = - \frac{1}{\tilde{\phi}^2} + \rho,
\eea
\bea
\tilde{\nabla}^2\tilde{\Phi} = - \half \rho,
\eea
and the force ratio
\bea
\frac{\left|\rbm{F}_{(\phi)} \right|}{\left|\rbm{F}_{(\Phi)} \right|} = \left( \frac{\qsubrm{M}{Pl}^2}{M}\right)^2\frac{\left| \tilde{\nabla}\tilde{\phi}\right|}{\left| \tilde{\nabla}\tilde{\Phi}\right|}
\eea
\section{Results}

\section{Discussion}


\end{document}
