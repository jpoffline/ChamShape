\documentclass[a4paper, 12pt]{article}
\NeedsTeXFormat{LaTeX2e}[1996/06/01]
\usepackage[usenames]{color} 
\usepackage{amsmath,amssymb}
\numberwithin{equation}{section}
\usepackage[top=2.5cm,left=3cm,right=3cm,bottom=3cm,foot=1cm,]{geometry}
%\linespread{1.5}
\usepackage[breaklinks, colorlinks, citecolor=blue]{hyperref}
\usepackage{multirow,array,graphicx}
\usepackage{sidecap,epstopdf,dcolumn}
\usepackage{bm}% bold math
\usepackage[font={small,it}]{caption}
\usepackage[footnotesize]{subfigure}
\usepackage[]{appendix}
%%%% Make citations use [1-5] rather than [1, 2, 3, 4, 5]
\usepackage[numbers, square, comma,  sort&compress]{natbib}
%\renewcommand{\cite}{\citep}
%%%% END
%\usepackage{ulem}
\setcounter{tocdepth}{3}
\setcounter{secnumdepth}{3}


\renewcommand{\sectionmark}[1]{\markright{\thesection.\ #1}{}}


\input{defs}


\begin{document}


\title{{\bf Shaping the chameleon}}
\author{\sc{Jonathan A. Pearson\footnote{E-mail: \href{mailto:j.pearson@nottingham.ac.uk}{j.pearson@nottingham.ac.uk}}}\\ \\ \it{School of Physics \& Astronomy} \\ \it{University of Nottingham} \\ \it{Nottingham, NG7 2RD}}

\date{\today}



\maketitle
\begin{abstract}
In conservation with Clare Burrage, Ed Copeland, James Stevenson, Adam Moss...
\end{abstract}
\clearpage

\tableofcontents   


\section{Introduction}
The idea is to understand the differences between the chameleon force and gravitational force for source objects with different shapes -- circles, ellipsoids, etc.
\section{The model}
In the static regime the chameleon scalar $\phi$ satisfies
\bse
\bea
\label{eq:cham-eom}
\nabla^2\phi = -\frac{\Lambda^5}{\phi^2} + \frac{\rho}{M},
\eea
and the gravitational potential $\Phi$ satisfies Laplace's equation,
\bea
\label{eq:laplaceseqn}
\nabla^2\Phi = - \rho
\eea
\ese
We compute the forces due to the chameleon and gravitational scalars by taking the gradient of the relevant scalar:
\bea
\label{eq:forces}
\rbm{F}_{(\phi)} = \nabla\phi,\qquad
\rbm{F}_{(\Phi)} = \nabla\Phi
\eea
\section{Numerical methods}
\begin{itemize}
\item Solve (\ref{eq:cham-eom}) via gradient flow; finite difference derivatives discretized to fourth order accuracy.
\item Solve (\ref{eq:laplaceseqn}) via SoR; discretized to second order accuracy.
\item The forces (\ref{eq:forces}) are computed with finite difference derivatives discretized to fourth order
\end{itemize}
\subsection{Simulated annealing}
We have a suggestion to obtain the optimal shape via simulated annealing. This would be implemented by first specifying some given matter distribution (e.g., a sphere), and computing the scalar and gravitational forces. A new shape is randomly proposed -- by switching ``on'' or ``off'' locations which are supposed to have matter. If the force discrepancies are greater for this new shape, it is kept, and the whole process is repeated until an optimal shape is obtained. One needs to be careful to only proposed connected objects.

This strategy is relatively straight-forward, but computationally very intensive -- one way to help is to parallelise the code.
\addcontentsline{toc}{section}{Acknowledgements}

\section*{Acknowledgements}

\appendix
\section{Schemes}

\textbf{\textit{SoR}} This is used to solve Laplace's equation. Fictitious ``time-steps'' are used, and indexed by $n$. Convergence is determined by the parameter $\omega$. The updating algorithm is
\bea
Q^{n+1}_{i,j} = (1-\omega) Q^{n}_{i,j} + \frac{\omega}{4}\left[ Q^n_{i+1,j} + Q^{n+1}_{i-1,j} + Q^n_{i,j+1} + Q^{n+1}_{i,j-1} + h^2\rho \right]
\eea

\textbf{\textit{Fourth order finite difference derivatives}}
\bea
\pd{Q}{x} \approx \frac{-Q_{i+2} + 8 Q_{i+1} - 8 Q_{i-1} + Q_{i-2}}{12h}
\eea
\bea
\pd{^2Q}{x^2} \approx \frac{-Q_{i+2} + 16 Q_{i+1} - 30 Q_i + 16 Q_{i-1} - Q_{i-2}}{12h^2}
\eea
\addcontentsline{toc}{section}{References}
\bibliographystyle{JHEP}
\footnotesize{
\bibliography{refs}
}
\end{document}
