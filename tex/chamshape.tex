\documentclass[a4paper, 12pt]{article}
\NeedsTeXFormat{LaTeX2e}[1996/06/01]
\usepackage[usenames]{color} 
\usepackage{amsmath,amssymb}
\numberwithin{equation}{section}
\usepackage[top=2.5cm,left=3cm,right=3cm,bottom=3cm,foot=1cm,]{geometry}
%\linespread{1.5}
\usepackage[breaklinks, colorlinks, citecolor=blue]{hyperref}
\usepackage{multirow,array,graphicx}
\usepackage{sidecap,epstopdf,dcolumn}
\usepackage{bm}% bold math
\usepackage[font={small,it}]{caption}
\usepackage[footnotesize]{subfigure}
\usepackage[]{appendix}
%%%% Make citations use [1-5] rather than [1, 2, 3, 4, 5]
\usepackage[numbers, square, comma,  sort&compress]{natbib}
%\renewcommand{\cite}{\citep}
%%%% END
%\usepackage{ulem}
\setcounter{tocdepth}{3}
\setcounter{secnumdepth}{3}
\newcommand{\note}[1]{{\color{blue}[#1]}}


\renewcommand{\sectionmark}[1]{\markright{\thesection.\ #1}{}}


\input{defs}


\begin{document}


\title{{\bf Shaping the chameleon}}
\author{\sc{Jonathan A. Pearson\footnote{E-mail: \href{mailto:j.pearson@nottingham.ac.uk}{j.pearson@nottingham.ac.uk}}}\\ \\ \it{School of Physics \& Astronomy} \\ \it{University of Nottingham} \\ \it{Nottingham, NG7 2RD}}

\date{\today}



\maketitle
\begin{abstract}
In conservation with Clare Burrage, Ed Copeland, James Stevenson, Adam Moss...
\end{abstract}
%\clearpage

\tableofcontents   


\section{Introduction}
The idea is to understand the differences between the chameleon force and gravitational force for source objects with different shapes -- circles, ellipsoids, etc.

The aim is to build on the analytic results and understanding developed by Burrage, Copeland, and Stevenson; they developed analytic solutions to the governing equations whilst using some simplifying (and justified) assumptions about the dominant contributions to the relevant equations. They obtained the fields surrounding an ellipsoid source. We intend to reproduce their results, and extend to other shapes. This necessitates the use of numerical methods.




\section{The model}
In the static regime the chameleon scalar $\phi$ satisfies
\bse
\bea
\label{eq:cham-eom}
\nabla^2\phi = -\frac{\Lambda^5}{\phi^2} + \frac{\rho}{M},
\eea
and the gravitational potential $\Phi$ satisfies Laplace's equation,
\bea
\label{eq:laplaceseqn}
\nabla^2\Phi = - \rho / M
\eea
\ese
The forces due to the chameleon and gravitational scalars are computed by taking the gradient of the relevant scalar:
\bea
\label{eq:forces}
\rbm{F}_{(\phi)} =- \nabla\phi,\qquad
\rbm{F}_{(\Phi)} = -\nabla\Phi.
\eea
We solve (\ref{eq:cham-eom}) using numerical methods outlined in Appendix \ref{sec:nummethods} for a given  source density function, $\rho(\rbm{x})$. 
\subsection{Specification of the source density profile}
We use a ``step-function'' to describe $\rho$, whereby the density has value $\rho_0$ in the interior of the source, and $\qsubrm{\rho}{bg}$ in the exterior of the source (this is supposed to represent the ambient, possibly cosmological, density). To be concrete, we set
\bea
\rho(\rbm{x}) = \left\{ \begin{array}{cc} \rho_0 & \mbox{inside}, \\ \qsubrm{\rho}{bg} & \mbox{outside}.\end{array}\right.
\eea
The spatial locations which are defined as ``inside'' and ``outside'' are those inside the given shape under consideration (circles, ellipses, etc).
\subsection{Boundary conditions and parameter values}
For boundary conditions we set 
\bea
\phi(\rbm{x}_{\infty}) = \qsubrm{\phi}{bg} = \sqrt{\frac{M\Lambda^5}{\qsubrm{\rho}{bg}}}.
\eea
This is the value of the scalar which minimizes the effective potential.

Typically we set
\bea
\label{paramvalues}
\Lambda = 1, \qquad M = 10^3, \qquad \rho_0 = 12, \qquad \qsubrm{\rho}{bg} = 0.1.
\eea
This is the  hierarchy of scales required to obtain the screening solutions of interest. Note that we require a large density contrast, $\delta \defn \frac{\rho_0}{\qsubrm{\rho}{bg}}-1$, and large $M$.
\section{Results in 2D}
\label{sec:resulys_2d}
Here we present results of numerically solving the equations in 2D;  the chameleon scalar $\phi$ satisfies
\bea
\pd{^2\phi}{x^2} + \pd{^2\phi}{y^2} = - \frac{\Lambda^5}{\phi^2} + \frac{\rho(x,y)}{M}.
\eea
\subsection{Circular sources}
Here we define the shape to be inside the circle
\bea
x^2 + y^2 =R^2.
\eea
See \fref{fig:circ} for a plot of the chameleon force; there is a shell of ``force'' around the object. The force increases linearly inside the object, and then dies off with some negative power in the exterior before rising again due to the  non-zero background density. In \fref{fig:circ-1dplots} we provide some more illuminating plots, showing the gravitational potential, and the forces generated by the circular source. One may be nervous about the form of the gravitational force: it increases with distance from the source. This corresponds exactly to the behavior we found from the analytic solution, which we plotted in \fref{fig:poisson_soln} in Appendix \ref{poissonseqsoln} (this is, in some sense, a feature which is present due to the non-zero background density).

\begin{figure}[!t]
      \begin{center}
{\includegraphics[scale=0.8,angle=0]{images/sp_lorho_force_final.pdf}}
      \end{center}
\caption{ The force density due to the chameleon around a circular source. This source has size $R=4$. The other parameter values are those given in (\ref{paramvalues}).   The colour scheme is such that the largest forces are the brightest colours, and any forces lower than half of the maximum are coloured in black.  The force increases linearly from the origin, until some maximum is reached at the objects surface; this is better shown in \fref{fig:circ-1dplots}(c).}\label{fig:circ}
\end{figure}

\begin{figure}[!t]
      \begin{center}
\subfigure[\,  Chameleon scalar  ]{\includegraphics[scale=0.5,angle=0]{images/circ/circ_fxy_cham_final}}
\subfigure[\,  Gravitational potential  ]{\includegraphics[scale=0.5,angle=0]{images/circ/circ_fxy_gpot_final}}
\subfigure[\,  Gravitational and chameleon forces (left), and ratios of the forces (right)  ]{\includegraphics[scale=1,angle=0]{images/circ/circ_fcomp_x_final}}
      \end{center}
\caption{ Plots of various fields for the same 2D circular source as given in \fref{fig:circ}. In (a) and (b) we plot the values of the chameleon scalar and gravitational potential as a function of distance from the centre of the circular source. One should note that the value of $\Phi(\mbox{origin})$ is arbitrary, since Poisson's equation is invariant under the shift $\Phi \rightarrow \Phi + c$, where $c$ is a constant. In (c) we plot the forces due to the gravitational and chameleon scalars. The rightmost panel in (c) shows the ratio of the chameleon and gravitational forces. }\label{fig:circ-1dplots}
\end{figure}

\subsection{Ellipsoidal sources}
Here we define the shape to be inside the ellipse
\bea
\label{ellipse-eq}
\left( \frac{x}{b}\right)^2 + \left( \frac{y}{a}\right)^2 = R^2
\eea
See \fref{fig:ellip} for a plot of the chameleon force density, and \fref{fig:el-1dplots} for comparisons of the gravitational and chameleon forces down the $x$- and $y$-axes.

\begin{figure}[!t]
      \begin{center}
{\includegraphics[scale=0.8,angle=0]{images/el_force_final}}
      \end{center}
\caption{ Force due to the chameleon around an ellipsoidal source. This source  has $a = 3.5, b = 1$, and $R=4$ as defined in (\ref{ellipse-eq}); the densities and all other parameters are the same as the source shown in \fref{fig:circ}. Here we observe that the force profiles along the $x$- and $y$-axes are rather different. The force at the object's surface at $y=0$ is lower than the force at the object's surface at $x=0$.  }\label{fig:ellip}
\end{figure}

\begin{figure}[!t]
      \begin{center}
\subfigure[\,  Down the $x$-axis  ]{\includegraphics[scale=1,angle=0]{images/el/el_fcomp_x_final}}
\subfigure[\,  Down the $y$-axis  ]{\includegraphics[scale=1,angle=0]{images/el/el_fcomp_y_final}}
      \end{center}
\caption{  Comparisons of the chameleon and gravitational forces down the $x$- and $y$-axes for the elliptical source shown in \fref{fig:ellip}. The maximum force acheived at the source's surface is largest down the $y$-axis for the chameleon force, and down the $x$-axis for the gravitational force. This is only very slight for the example shown here, but highlights an important distinction between the nature of the gravitational and chameleonic forces (this also corroborates one of the main results of Stevenson et al). The right-most panels  shows the ratio of the forces due to the gravitational and chamelon scalars. }\label{fig:el-1dplots}
\end{figure}

\subsection{Other shapes}
\fref{fig:varioushapes}.

\begin{figure}[!t]
      \begin{center}
\subfigure[\, Crossed ellipses ]{\includegraphics[scale=0.6,angle=0]{images/ep2_lorho_force_final.pdf}}
      \end{center}
\caption{ The force due to the chameleon around various sources. In (a) there are two ellipses at right-angles each with the same properties as those in \fref{fig:ellip}.  \note{replace with converged versions}}\label{fig:varioushapes}
\end{figure}

\section{Using the \textit{cartoon} method to shape axial chameleons in 3D}
The codes used in Section \ref{sec:resulys_2d} use gradient flow to obtain the shape of the chameleon fields in 2D. This is an initial choice, born out of numerical simplicity (as well as being able to easily see everything that is going on). However, as discussed in Appendix \ref{poissonseqsoln}, the gravitational potential in 2D and 3D are rather different in nature: and so we want to be able to test whether force discrepancies in 2D are due to the chameleon, or because we are using 2D. 

It is now to be remarked that whilst the numerical algorithms are simple to extend from 2D to 3D, the increase in computational resources is substantial. We can make progress if we assume that the source in 3D is axially symmetric. This allows us to obtain the chameleon profiles in a 2D slice of the object: this slice must be rotated forwards and backwards to capture the 3D-ness. One can also impose some periodic breaking of axial symmetry. This is the essence of the \textit{cartoon} method \cite{Alcubierre:1999ab}. 

The idea is to perform gradient flow in Cartesian coordinates in the $(x,y,z) = (x,0,z)$-plane, and then extrapolate the fields into the $y = \pm \Delta x$ planes using the assumed axial symmetry. This strategy has been sucessfully used in the construction of axially symmetric solitons, specifically cosmic vortons  \cite{Battye:2008mm}.

\section{Discussion}

\addcontentsline{toc}{section}{Acknowledgements}

\section*{Acknowledgements}

\appendix
\section{Numerical methods}
\label{sec:nummethods}
In summary, our numerical methodology is:
\begin{itemize}
\item Solve for the chameleon scalar (\ref{eq:cham-eom}) via gradient flow; finite difference derivatives discretized to fourth order accuracy.
\item Solve Poisson's equation (\ref{eq:laplaceseqn}) via SoR; discretized to second order accuracy.
\item The forces (\ref{eq:forces}) are computed with finite difference derivatives discretized to fourth order
\end{itemize}
In the remainder of this Appendix we shall explain the implementation of each of these methods.

\subsection{Gradient flow}
The scalar field profile is obtained by solving the gradient flow equation:
\bea
\label{gfeqn}
\dot{\phi} = \nabla^2\phi - \frac{\dd\qsubrm{V}{eff}}{\dd\phi}.
\eea
We discretize the time derivative in a very simple manner,
\bea
\dot{\phi} \approx \frac{\phi(t+1) - \phi(t)}{h_t}.
\eea
This can be used on the LHS of (\ref{gfeqn}) to yield an updating algorithm for the scalar; obviously, we also need a scheme to approximate the Laplacian: that is discussed in the next section. Note that we use a ``fictitious time-step'', $h_t$: this must be chosen to be much smaller than any  step-sizes in the spatial dimensions (else the problem is badly defined numerically).

A ``solution'' to the static problem is obtained when $\dot{\phi}=0$. This is quantified by computing an error estimate
\bea
E_{\phi} \defn \int \dd^2x\, \dot{\phi},
\eea
and waiting until $E_{\phi}\ll \varepsilon$, where $\epsilon$ is supposed to be a very small number, before a solution is declared to have been found. We are also estimating the error on a solution by computing the evolution of the integral of the modulus of the force,
\bea
\qsuprm{E}{F}_{\phi} \defn \int \dd^2x\, \left| \rbm{F}_{(\phi)}\right|;
\eea
this quantity should be independant of time, and so we wait until $\dot{E}\qsuprm{}{F}_{\phi}\ll \varepsilon$ before declaring that a solution has been found.



\subsection{Discretization schemes}
Here we explain the discretization schemes used. Space is discretized onto a lattice with step-size $h$.

\subsubsection{Fourth order finite difference derivatives}
The first and second derivatives of a quantity $Q$, say, are approximated at a given location via a fourth-order accurate finite difference scheme according to
\bea
\pd{Q}{x} \approx \frac{-Q_{i+2} + 8 Q_{i+1} - 8 Q_{i-1} + Q_{i-2}}{12h},
\eea
\bea
\pd{^2Q}{x^2} \approx \frac{-Q_{i+2} + 16 Q_{i+1} - 30 Q_i + 16 Q_{i-1} - Q_{i-2}}{12h^2}.
\eea
There are obvious equivalents for the $y$-derivatives; these can be used to compute the forces and relevant Laplacians.

\subsubsection{SoR} 
This is used to solve Laplace's equation. Fictitious ``time-steps'' are used, and indexed by $n$. Convergence is determined by the SoR parameter, $\omega = \frac{2}{1+ {\pi}/{N}}$ where $N$ is the number of grid-points in a given direction. A quantity $Q$ is discretized as $Q^n_{i,j}$ corresponding to the value of $Q$ at lattice site $(i,j)$, and `time-step'' $n$. The updating algorithm is
\bea
Q^{n+1}_{i,j} = (1-\omega) Q^{n}_{i,j} + \frac{\omega}{4}\left[ Q^n_{i+1,j} + Q^{n+1}_{i-1,j} + Q^n_{i,j+1} + Q^{n+1}_{i,j-1} + h^2\rho \right],
\eea
where we remind that $h$ is the space step-size.

\subsection{Simulated annealing to find the optimum shape}
We have a suggestion to obtain the optimal shape via simulated annealing. This would be implemented by first specifying some given matter distribution (e.g., a sphere), and computing the scalar and gravitational forces. A new shape is randomly proposed -- by switching ``on'' or ``off'' locations which are supposed to have matter. If the force discrepancies are greater for this new shape, it is kept, and the whole process is repeated until an optimal shape is obtained. One needs to be careful to only proposed connected objects.

This strategy is relatively straight-forward, but computationally very intensive -- one way to help is to parallelise the code.
\section{Solving Poisson's equation for symmetric sources}
\label{poissonseqsoln}
Here we solve Poisson's equation, 
\bea
\nabla^2\Phi = - \rho /M,
\eea
for a source with constant density $\rho_0$ immersed in a medium with density $\qsubrm{\rho}{bg}$; the source has radius $R$.
\subsection{Poisson's equation in 2D}

Here we assume a 2D  circular source. Due to the symmetry of the source, the gravitational potential $\Phi = \Phi(r)$ only. Note that the Laplacian in 2D polars is
\bea
\nabla^2\Phi = \frac{1}{r}\frac{\dd}{\dd r}\left( r \frac{\dd\Phi}{\dd r}\right).
\eea
The solution is
\bea
\Phi(r) = -\frac{\rho}{4M}r^2 + A\ln r + B,
\eea
where $A$ and $B$ are integration constants; their values will be fixed by requiring continuity, smoothness, and regularity in various regimes. The radial component of the force (the only component, infact) is
\bea
F_r(r) =  \frac{\rho}{2M}r - \frac{A}{r}.
\eea
From hereon we will drop the ``$r$'' index from the force component.
We are interested in obtaining the gravitational potential surrounding and within a circle of radius $R$, density $\rho_0$, immersed in a background density $\qsubrm{\rho}{bg}$. One requires $\qsubrm{A}{in}=0$ for regularity at the origin. By matching the forces at the surface $r=R$ one obtains the   expression for $\qsubrm{A}{out}$,
\bse
\bea
\qsubrm{A}{out} = \frac{ \qsubrm{\rho}{bg}R^2}{2M}\left( 1 - \frac{\rho_0}{\qsubrm{\rho}{bg}}\right).
\eea
By matching the potentials, one also obtains
\bea
\qsubrm{B}{bg} = B_0 + \frac{R^2\qsubrm{\rho}{bg}}{4M}\left( 1- \frac{\rho_0}{\qsubrm{\rho}{bg}}\right)\left( 1 - 2\ln R\right).
\eea
\ese


The potential inside and outside the object is given by
\bse
\label{poiss-soln-1}
\bea
\qsubrm{\Phi}{in}(r) = - \frac{\rho_0}{4M}r^2 + B_0,
\eea
\bea
\qsubrm{\Phi}{out}(r) = - \frac{\qsubrm{\rho}{bg}}{4M}r^2 +  \frac{ \qsubrm{\rho}{bg}R^2}{2M}\left( 1 - \tfrac{\rho_0}{\qsubrm{\rho}{bg}}\right)\ln r + B_0+ \frac{R^2\qsubrm{\rho}{bg}}{4M}\left( 1- \tfrac{\rho_0}{\qsubrm{\rho}{bg}}\right)\left( 1 - 2\ln R\right),\nonumber\\
\eea
\ese
and the force inside and outside are given by
\bse
\label{poiss-soln-2}
\bea
\qsubrm{F}{in}(r) = \frac{\rho_0}{2M}r,
\eea
\bea
\qsubrm{F}{out}(r) =  \frac{\qsubrm{\rho}{bg}}{2M}r -   \frac{\qsubrm{\rho}{bg}R^2}{2M}{}{ }\left( 1 -\tfrac{ \rho_0}{\qsubrm{\rho}{bg}} \right) \frac{1}{r} .
\eea
\ese
See \fref{fig:poisson_soln} for a plot of these solutions.
\subsection{Poisson's equation in 3D}
In 3D, for a spherically symmetric gravitational potential, the Laplacian reads
\bea
\nabla^2\Phi = \frac{1}{r^2}\frac{\dd}{\dd r}\left( r^2 \frac{\dd\Phi}{\dd r}\right).
\eea
The solution to Poisson's equation (again, with a constant source density $\rho$) is 
\bea
\Phi(r) = - \frac{\rho}{6M} r^2- \frac{A}{r}+ B,
\eea
in terms of two integration constants $A$ and $B$.   After requiring  continuity and regularity, the potential and force inside and out are given by
\bse
\label{poiss-soln-3}
\bea
\qsubrm{\Phi}{in}(r) = - \frac{\qsubrm{\rho}{in}}{6M} r^2 + \qsubrm{B}{in},
\eea
\bea
\qsubrm{\Phi}{out}(r) = - \frac{\qsubrm{\rho}{out}}{6M} r^2+ \frac{\qsubrm{\rho}{out}R^3}{3M} \left(\tfrac{\qsubrm{\rho}{in} }{\qsubrm{\rho}{out}} -1\right) \frac{1}{r}+\frac{\qsubrm{\rho}{out}R^2}{2M} \left(1 - \tfrac{\qsubrm{\rho}{in}}{\qsubrm{\rho}{out}}\right)  +\qsubrm{B}{in},
\eea
\ese
\bse
\label{poiss-soln-4}
\bea
\qsubrm{F}{in} (r)=  \frac{\qsubrm{\rho}{in}}{3M} r,
\eea
\bea
\qsubrm{F}{out}(r) = \frac{\qsubrm{\rho}{out}}{3M} r+\frac{\qsubrm{\rho}{out}R^3}{3M} \left(\tfrac{\qsubrm{\rho}{in} }{\qsubrm{\rho}{out}}-1 \right)  \frac{1}{r^2} .
\eea
\ese
See \fref{fig:poisson_soln} for a plot of these solutions.

\begin{figure}[!t]
      \begin{center}
{\includegraphics[scale=1,angle=0]{images/rhoin_12_comps}}
      \end{center}
\caption{ The gravitational potential (left) and force (right) for 2D and 3D circularly and spherically symmetric sources with constant density. These correspond to the solutions we presented in Appendix \ref{poissonseqsoln}. The values of the parameters are exactly those we gave in (\ref{paramvalues}), along with $R=4$ for the radius of the source. }\label{fig:poisson_soln}
\end{figure}

\addcontentsline{toc}{section}{References}
\bibliographystyle{JHEP}
\footnotesize{
\bibliography{refs}
}
\end{document}
