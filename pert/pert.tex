%\documentclass[a4paper,eqsecnum,twoside,aps,floatfix,notitlepage]{revtex4}

\documentclass[amsmath,amssymb,10pt,eqsecnum]{revtex4}
\usepackage{graphicx,float,subfloat}
\usepackage[breaklinks, colorlinks, citecolor=blue]{hyperref}
\usepackage{bm}% bold math
\linespread{1}
\usepackage{subfigure}
\usepackage{hyperref}

\newcommand{\Planck}{{\textit{Planck }}}
\newcommand{\SPT}{{\textit{SPT }}}
\newcommand{\CFHTLenS}{{\textit{CFHTLenS }}}
\newcommand{\CoRE}{{\textit{CoRE }}}
\newcommand{\WMAP}{{\textit{WMAP }}}
\newcommand{\Euclid}{{{\it Euclid}}}
\newcommand{\CAMB}{{\tt{CAMB }}}
\newcommand{\MGCAMB}{{\tt{MGCAMB }}}

%\newcommand{\vphi}[0]{{\color{green}{\delta\phi}}}
%\newcommand{\dlg}[0]{{\color{red}{\lp g}}}
\newcommand{\vphi}[0]{\delta\phi}
\newcommand{\lpa}[0]{\lp A}
\newcommand{\dlg}[0]{\lp g}

\newcommand{\depg}[0]{\ep g}
\newcommand{\virt}[0]{\hat{\delta}}
\newcommand{\Dvphi}[1]{\nabla_{#1}\vphi}
\newcommand{\DDvphi}[2]{\nabla_{#1}\nabla_{#2}\vphi}

\newcommand{\Dvvphi}[1]{\nabla_{#1}\virt\vphi}
\newcommand{\DDvvphi}[2]{\nabla_{#1}\nabla_{#2}\virt\vphi}

\newcommand{\tmat}[4]{\left( \begin{array}{cc} #1 & #2 \\ #3 & #4 \end{array}\right)}
\newcommand{\expec}[1]{\left\langle #1\right\rangle}
\newcommand{\half}[0]{\frac{1}{2}}
\newcommand{\cs}[3]{\Gamma^{#1}_{\,\,\,\, #2#3}}

\newcommand{\pd}[2]{\frac{\partial #1}{\partial #2}}
\newcommand{\ld}[0]{\mathcal{L}}
\newcommand{\md}[0]{\mathcal{M}}
\newcommand{\dd}[0]{\textrm{d}}
\newcommand{\defn}[0]{\equiv}
\newcommand{\diag}[0]{\textrm{diag}}
\newcommand{\qsubrm}[2]{{#1}_{\scriptsize{\textrm{#2}}}}
\newcommand{\qsuprm}[2]{{#1}^{\scriptsize{\textrm{#2}}}}
\newcommand{\qsubprm}[3]{{#1}^{\scriptsize{\textrm{#2}}}_{\scriptsize{\textrm{#3}}}}
 \newcommand{\subsm}[2]{{#1}_{\scriptscriptstyle{#2}}}

\newcommand{\supsm}[2]{{#1}^{\scriptscriptstyle{#2}}}
\newcommand{\symmb}[0]{\varrho}
\newcommand{\AW}[0]{A_{\mathcal{W}}}
\newcommand{\BW}[0]{B_{\mathcal{W}}}
\newcommand{\CW}[0]{C_{\mathcal{W}}}
\newcommand{\DW}[0]{D_{\mathcal{W}}}
\newcommand{\EW}[0]{E_{\mathcal{W}}}

\newcommand{\AP}[0]{A_{\mathcal{P}}}
\newcommand{\BP}[0]{B_{\mathcal{P}}}
\newcommand{\CP}[0]{C_{\mathcal{P}}}
\newcommand{\DP}[0]{D_{\mathcal{P}}}
\newcommand{\EP}[0]{E_{\mathcal{P}}}
\newcommand{\FP}[0]{F_{\mathcal{P}}}
\newcommand{\GP}[0]{G_{\mathcal{P}}}

\newcommand{\sol}[0]{\ld_{\scriptscriptstyle\{2\}}}

\newcommand{\tis}[0]{ {\theta}^{\scriptscriptstyle\rm{S}}}
\newcommand{\tisdot}[0]{ {\dot{\theta}}^{\scriptscriptstyle\rm{S}}}
\newcommand{\pis}[0]{ {\Pi}^{\scriptscriptstyle\rm{S}}}
\newcommand{\xis}[0]{ {\xi}^{\scriptscriptstyle\rm{S}}}
\newcommand{\xisdot}[0]{ {\dot{\xi}}^{\scriptscriptstyle\rm{S}}}
\newcommand{\xisddot}[0]{ {\ddot{\xi}}^{\scriptscriptstyle\rm{S}}}
\newcommand{\xisdddot}[0]{ {\dddot{\xi}}^{\scriptscriptstyle\rm{S}}}
\newcommand{\lcdm}[0]{$\Lambda$CDM}

\newcommand{\coup}[0]{\mathcal{Q}}
\newcommand{\vmkin}[0]{\mathcal{K}}



\newcommand{\newchap}[2]{\chapter{#1}
\markboth{\MakeUppercase{Chapter \thechapter.\ #2}}{}}
\newcommand{\nphiu}[1]{\nabla^{#1}\phi}
\newcommand{\nphid}[1]{\nabla_{#1}\phi}

\newcommand{\gbm}[1]{\bm{#1}}
\newcommand{\rbm}[1]{{\bf{#1}}}
\newcommand{\ci}[0]{\textrm{i}}
%\newcolumntype{V}{>{\centering\arraybackslash} m{.4\linewidth} }
\newcommand{\kin}[0]{{\mathcal{X}}}
\newcommand{\hct}[0]{\mathcal{H}}
\renewcommand{\figurename}{Figure}
\newcommand{\ep}[0]{{ {\delta}_{\scriptscriptstyle{\rm{E}}}}}
\newcommand{\lp}[0]{{ {\delta}_{\scriptscriptstyle{\rm{L}}}}}
\def\be{\begin{equation}}
\def\ee{\end{equation}}
\def\bea{\begin{eqnarray}}
\def\eea{\end{eqnarray}}
\def\bse{\begin{subequations}}
\def\ese{\end{subequations}}
\newcommand{\lied}[1]{\pounds_{#1}}

\newcommand{\sech}[0]{\textrm{ sech}}
% Planck style says this should be Fig. 1
\newcommand{\fref}[1]{{Fig.~\ref{#1}}}
\newcommand{\tref}[1]{{Table \ref{#1}}}
\newcommand{\secref}[1]{{section \ref{#1}}}
\newcommand{\Secref}[1]{{Section \ref{#1}}}
% PUT CATCH ON THE END OF SQUARE-ROOT SYMBOLS
\newcommand{\rsbb}[2]{#1_{\mathbb{#2}}}

\newcommand{\dg}[0]{\delta g}
\let\oldsqrt\sqrt
% it defines the new \sqrt in terms of the old one
\def\sqrt{\mathpalette\DHLhksqrt}
\def\DHLhksqrt#1#2{%
\setbox0=\hbox{$#1\oldsqrt{#2\,}$}\dimen0=\ht0
\advance\dimen0-0.2\ht0
\setbox2=\hbox{\vrule height\ht0 depth -\dimen0}%
{\box0\lower0.4pt\box2}}

\newcommand{\sbm}[2]{#1_{\mathbb{#2}}}

\newcommand{\comment}[1]{{\color{red}[#1]}}

 
\begin{document}

\title{Shaping the chameleon}
\author{Jonathan A. Pearson}
\email{j.pearson@nottingham.ac.uk}
\affiliation{School of Physics \& Astronomy, University of Nottingham, Nottingham, NG7 2RD, U.K.}	
\date{\today}

\begin{abstract}
In conversation with C. Burrage
\end{abstract}

\maketitle


\tableofcontents

 
 
 
\section{Introduction} 


\bse
\bea
\phi = \left\{\begin{array}{ccc}
\qsubrm{\phi}{c} & & 0 < r < S,\\
\frac{\rho r^2}{6m} + \frac{c}{r} + d && S < r < R,\\
\frac{\alpha}{r} + \phi_{\infty} && R < r
\end{array}\right.
\eea
\ese
Matching value and derivative at $r = S$:
\bea
\qsubrm{\phi}{c} = \frac{\rho S^2}{6m} + \frac{c}{S} + d ,\qquad 0 = \frac{\rho S}{3m} - \frac{c}{S^2}
\eea
Matching value and derivative at $r = R$:
\bea
\frac{\rho R^2}{6m} + \frac{c}{R} + d = \frac{\alpha}{R} + \phi_{\infty},\qquad \frac{\rho R}{3m} - \frac{c}{R^2} = - \frac{\alpha}{R^2}.
\eea
We thus combine to obtain
\bea
c = \frac{\rho S^3}{3m},\qquad \alpha = \frac{\rho}{3m} \left( S^3 - R^3\right),\qquad d = \qsubrm{\phi}{c} - \frac{\rho S^2}{2m}.
\eea
Now, let all quantities aquire an angular perturbation,
\bea
X \rightarrow X_0 + \varepsilon X_1(\theta).
\eea
Collecting like powers of $\varepsilon$ yields
\bea
\frac{\alpha_1}{\alpha_0} = \frac{\rho_1}{\rho_0} + \frac{3}{1 - \left(S_0/R_0\right)^3} \left[ \frac{R_1}{R_0} - \left( \frac{S_0}{R_0}\right)^2 \frac{S_1}{R_0} \right].
\eea
Note that $\alpha$ will be the magnitude of the radial component of the chameleon force outside the source. Note that if $S_0 \approx R_0$ then the denominator above will become very small, meaning that $\alpha_1/\alpha_0$ could become rather large.

The volume of the thin shell is
\bea
\mathcal{V} = \frac{4}{3}\pi \left( R^3 - S^3\right).
\eea
Hence,
\bea
\mathcal{V} = \mathcal{V}_0 + 4 \pi R_0^3\left[ \frac{R_1}{R_0} - \left( \frac{S_0}{R_0}\right)^2 \frac{S_1}{R_0} \right]
\eea
{\tt Does this make sense? volume is an integrated quantity, and here we have angular-dependent terms on the RHS}
\end{document}
