%\documentclass[a4paper,eqsecnum,twoside,aps,floatfix,notitlepage]{revtex4}

\documentclass[amsmath,amssymb,10pt,eqsecnum]{revtex4}
\usepackage{graphicx,float,subfloat}
\usepackage[breaklinks, colorlinks, citecolor=blue]{hyperref}
\usepackage{bm}% bold math
\linespread{1}
\usepackage{subfigure}
\usepackage{hyperref}

\input{defs}
 
\begin{document}

\title{Shaping the chameleon}
\author{Jonathan A. Pearson}
\email{j.pearson@nottingham.ac.uk}
\affiliation{School of Physics \& Astronomy, University of Nottingham, Nottingham, NG7 2RD, U.K.}	
\date{\today}

\begin{abstract}
In conversation with C. Burrage
\end{abstract}

\maketitle


\tableofcontents

 
 
 
\section{Introduction} 


\bse
\bea
\phi = \left\{\begin{array}{ccc}
\qsubrm{\phi}{c} & & 0 < r < S,\\
\frac{\rho r^2}{6m} + \frac{c}{r} + d && S < r < R,\\
\frac{\alpha}{r} + \phi_{\infty} && R < r
\end{array}\right.
\eea
\ese
Matching value and derivative at $r = S$:
\bea
\qsubrm{\phi}{c} = \frac{\rho S^2}{6m} + \frac{c}{S} + d ,\qquad 0 = \frac{\rho S}{3m} - \frac{c}{S^2}
\eea
Matching value and derivative at $r = R$:
\bea
\frac{\rho R^2}{6m} + \frac{c}{R} + d = \frac{\alpha}{R} + \phi_{\infty},\qquad \frac{\rho R}{3m} - \frac{c}{R^2} = - \frac{\alpha}{R^2}.
\eea
We thus combine to obtain
\bea
c = \frac{\rho S^3}{3m},\qquad \alpha = \frac{\rho}{3m} \left( S^3 - R^3\right),\qquad d = \qsubrm{\phi}{c} - \frac{\rho S^2}{2m}.
\eea
Now, let all quantities aquire an angular perturbation,
\bea
X \rightarrow X_0 + \varepsilon X_1(\theta).
\eea
Collecting like powers of $\varepsilon$ yields
\bea
\frac{\alpha_1}{\alpha_0} = \frac{\rho_1}{\rho_0} + \frac{3}{1 - \left(S_0/R_0\right)^3} \left[ \frac{R_1}{R_0} - \left( \frac{S_0}{R_0}\right)^2 \frac{S_1}{R_0} \right].
\eea
Note that $\alpha$ will be the magnitude of the radial component of the chameleon force outside the source. Note that if $S_0 \approx R_0$ then the denominator above will become very small, meaning that $\alpha_1/\alpha_0$ could become rather large.

In spherical polar coordinates $x^i = (r, \theta, \chi)$, the force is
\bea
\rbm{F} = - \nabla \phi = - \left( \pd{\phi}{r} \hat{\rbm{r}} + \frac{1}{r}\pd{\phi}{\theta}\hat{\gbm{\theta}} +\frac{1}{r\sin\theta} \pd{\phi}{\chi} \hat{\gbm{\chi}} \right).
\eea
Write $F = |\rbm{F}|$. Then,
\bea
F^2 = F_r^2 + F_{\theta}^2 + F_{\chi}^2,\qquad F_r \defn \pd{\phi}{r},\qquad F_{\theta} \defn \frac{1}{r}\pd{\phi}{\theta},\qquad F_{\chi} \defn \frac{1}{r\sin\theta}\pd{\phi}{\chi}.
\eea
Since we have $\alpha = \alpha_0 + \varepsilon\alpha_1$, we have the corresponding decomposition of $\phi$:
\bea
\phi = \phi_{\infty} + \frac{\alpha_0 + \varepsilon \alpha_1(\theta, \chi)}{r} .
\eea
Hence, writing $\phi = \phi_0 + \varepsilon\phi_1$, we identify
\bea
\phi_0 = \phi_{\infty} + \frac{\alpha_0}{r},\qquad \phi_1 = \frac{\alpha_1(\theta, \chi)}{r}.
\eea
The only non-zero contribution to the force at zeroth-order in $\varepsilon$ is 
\bea
F_r^{(0)} = \pd{\phi_0}{r} = - \frac{\alpha_0}{r^2}.
\eea
All contributions to the force are non-zero at first order in $\varepsilon$; they are
\bea
F_r^{(1)} = \pd{\phi_1}{r} = - \frac{\alpha_1}{r^2},\qquad F_{\theta}^{(1)} = \frac{1}{r^2}\pd{\alpha_1}{\theta},\qquad F_{\chi}^{(1)} = \frac{1}{r^2\sin\theta}\pd{\alpha_1}{\chi}.
\eea
Now,
\bea
\pd{\alpha_1}{\psi} = \alpha_0\left( \frac{1}{\rho_0}\pd{\rho_1}{\psi} +  \frac{3}{1 - \left(S_0/R_0\right)^3} \left[ \frac{1}{R_0} \pd{R_1}{\psi}- \left( \frac{S_0}{R_0}\right)^2 \frac{1}{R_0}\pd{S_1}{\psi}\right]\right),\qquad \psi \in \{\theta, \chi\}.
\eea
%The volume of the thin shell is
%\bea
%\mathcal{V} = \frac{4}{3}\pi \left( R^3 - S^3\right).
%\eea
%Hence,
%\bea
%\mathcal{V} = \mathcal{V}_0 + 4 \pi R_0^3\left[ \frac{R_1}{R_0} - \left( \frac{S_0}{R_0}\right)^2 \frac{S_1}{R_0} \right]
%\eea
%{\tt Does this make sense? volume is an integrated quantity, and here we have angular-dependent terms on the RHS}
\end{document}
